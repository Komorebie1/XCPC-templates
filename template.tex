\documentclass[10pt,a4paper]{article}
%\usepackage{zh_CN-Adobefonts_external}
\usepackage{xeCJK}
\usepackage{amsmath, amsthm}
\usepackage{listings,xcolor}
\usepackage{geometry} % 设置页边距
\usepackage{fontspec}
\usepackage{graphicx}
\usepackage[hidelinks]{hyperref}
\usepackage{setspace}
\usepackage{fancyhdr} % 自定义页眉页脚
\usepackage[backend=biber]{biblatex}
\usepackage{amssymb}
\usepackage{booktabs}

\setsansfont{Consolas} % 设置英文字体
\setmonofont[Mapping={}]{Consolas} % 英文引号之类的正常显示,相当于设置英文字体

\linespread{1.2}

\title{Template For ICPC}
\author{ZhengZX @ UESTC}
\definecolor{dkgreen}{rgb}{0,0.6,0}
\definecolor{gray}{rgb}{0.5,0.5,0.5}
\definecolor{mauve}{rgb}{0.58,0,0.82}

\pagestyle{fancy}

\lhead{\CJKfamily{kai} University of Electronic Science and Technology of China} %以下分别为左中右的页眉和页脚
\chead{}

\rhead{\CJKfamily{kai} 第 \thepage 页}
\lfoot{} 
\cfoot{\thepage}
\rfoot{}
\renewcommand{\headrulewidth}{0.4pt} 
\renewcommand{\footrulewidth}{0.4pt}
%\geometry{left=2.5cm,right=3cm,top=2.5cm,bottom=2.5cm} % 页边距
\geometry{left=3.18cm,right=3.18cm,top=2.54cm,bottom=2.54cm}
\setlength{\columnsep}{30pt}

\makeatletter

\makeatother



\lstset{
    language    = c++,
    numbers     = left,
    numberstyle={                               % 设置行号格式
        \small
        \color{black}
        \fontspec{Consolas}
    },
	commentstyle = \color[RGB]{0,128,0}\bfseries, %代码注释的颜色
	keywordstyle={                              % 设置关键字格式
        \color[RGB]{40,40,255}
        \fontspec{Consolas Bold}
        \bfseries
    },
	stringstyle={                               % 设置字符串格式
        \color[RGB]{128,0,0}
        \fontspec{Consolas}
        \bfseries
    },
	basicstyle={                                % 设置代码格式
        \fontspec{Consolas}
        \small\ttfamily
    },
	emphstyle=\color[RGB]{112,64,160},          % 设置强调字格式
    breaklines=true,                            % 设置自动换行
    tabsize     = 4,
    frame       = single,%主题
    columns     = fullflexible,
    rulesepcolor = \color{red!20!green!20!blue!20}, %设置边框的颜色
    showstringspaces = false, %不显示代码字符串中间的空格标记
	escapeinside={\%*}{*)},
}

\begin{document}
\title{XCPC Templates}
\author {Komorebie}
\maketitle
\tableofcontents
\newpage
\section{杂项}
\subsection{快读}
\lstinputlisting{杂项/快读.cpp}
\subsection{\_\_int128输出流自定义}
\lstinputlisting{杂项/__int128输出流.cpp}
\subsection{unordered\_map使用pair作为key}
\begin{spacing}{1.5}
对于任意结构体使用哈希,要先重载等于号(冲突时),然后在哈希函数中将所有的哈希值异或起来返回即可。
\end{spacing}
\lstinputlisting{杂项/hashtable.cpp}
\subsection{cout设置精度}
\lstinputlisting{杂项/cout设置精度.cpp}
\subsection{对拍相关}
\subsubsection{bat}
\lstinputlisting{杂项/对拍相关/bat.cpp}
\subsubsection{数据生成}
\lstinputlisting{杂项/对拍相关/数据生成.cpp}
\subsubsection{文件输入输出}
\lstinputlisting{杂项/对拍相关/文件输入输出.cpp}
\subsection{库函数}
\subsubsection{位运算函数}
\lstinputlisting{杂项/库函数/__builtin.cpp}
\subsubsection{批量递增赋值函数}
\lstinputlisting{杂项/库函数/批量递增赋值函数.cpp}
\subsubsection{数组随机打乱}
\lstinputlisting{杂项/库函数/数组随机打乱.cpp}
\subsection{字符串转化}
\subsubsection{数字转字符串}
\begin{spacing}{1.5}
to\_string函数会直接将你的各种类型的数字转换为字符串。
【不建议使用】itoa允许你将整数转换成任意进制的字符串,参数为待转换整数、目标字符数组、进制。但是其不是标准的C函数,且为Windows独有,且不支持 long long ,建议手写。
\end{spacing}
\lstinputlisting{杂项/字符串转化/数字转字符串.cpp}
\subsubsection{字符串转数字}
\lstinputlisting{杂项/字符串转化/字符串转数字.cpp}
\section{数据结构}
\subsection{并查集}
\begin{spacing}{1.5}
并查集究极版(支持维护size和到祖先的距离dis)
\end{spacing}
\lstinputlisting{数据结构/DSU.cpp}
\subsection{树状数组}
\lstinputlisting{数据结构/Fenwick.cpp}
\subsection{ST表}
\lstinputlisting{数据结构/ST.cpp}
\subsection{线段树}
\begin{spacing}{1.5}
支持区间修改和查询区间和(懒标记)。
\end{spacing}
\lstinputlisting{数据结构/SegmentTree.cpp}
\subsection{主席树}
\begin{spacing}{1.5}
注意先离散化再将离散化后的值逐个插入 (modify)。
例题: \href{https://www.luogu.com.cn/problem/P3834}{luguo P3834}
\begin{itemize}
    \item 给定 n 个整数构成的序列 a,将对于指定的闭区间 [l,r] 查询其区间内的第 k 小值。
    \item 满足 $1 \leq n,m \leq 2\times 10^5$,$0\le a_i \leq 10^9$,$1 \leq l \leq r \leq n$,$1 \leq k \leq r - l + 1$。
\end{itemize}
\end{spacing}
\lstinputlisting{数据结构/PresidentTree.cpp}
\subsection{树上启发式合并}
\lstinputlisting{数据结构/DSUonTree.cpp}
\subsection{莫队算法}
\begin{spacing}{1.5}
一定要先拓展区间再收缩区间,防止区间减到0及以下。
\end{spacing}
\lstinputlisting{数据结构/莫队.cpp}
\section{字符串}
\subsection{KMP}
\lstinputlisting{字符串/KMP.cpp}
\subsection{Manacher}
\lstinputlisting{字符串/Manacher.cpp}
\subsection{trie树}
\lstinputlisting{字符串/trie树.cpp}
\subsection{Z函数}
\begin{spacing}{1.5}
$z[i] = lcp(s[1:n-1], s[i: n-1])$
\end{spacing}
\lstinputlisting{字符串/Z函数.cpp}
\subsection{AC自动机}
\lstinputlisting{字符串/AC自动机.cpp}
\subsection{回文自动机}
\begin{spacing}{1.5}
应用
\begin{itemize}
    \item 本质不同回文子串个数
    \item 回文子串出现次数
\end{itemize}
每个节点代表一个回文串,每个节点的len表示回文串的长度,num表示回文串出现次数。
0号节点表示长度为0的回文串,1号节点表示长度为-1的回文串。
\end{spacing}
\lstinputlisting{字符串/回文自动机.cpp}
\subsection{后缀排序}
\begin{spacing}{1.5}
$sa$ 表示排第 $i$ 的后缀的前一半是第几个后缀, $sa2$ 表示排第$i$的后缀的后一半是第几个后缀,$rk$ 表示第 $i$ 个后缀排第几, 有$sa[rk[i]]=i, rk[sa[i]]=i$. 

$height[i]: lcp(sa[i],sa[i−1]) $,即排名为 $i$ 的后缀与排名为 $i−1$ 的后缀的最长公共前缀。

$height[rak[i]]$,即 $i$​ 号后缀与它前一名的后缀的最长公共前缀。

经典应用:
\begin{itemize}
    \item 两个后缀的最大公共前缀:$lcp(x,y)=min(heigh[x-y])$,用RMQ维护,$O(1)$ 查询。
    \item 可重叠最长重复子串:$height$ 数组中最大值
    \item 本质不同的字串的数量:枚举每一个后缀,第 $i$ 个后缀对答案的贡献为 $len-sa[i]+1-height[i]$
\end{itemize}
\end{spacing}
\lstinputlisting{字符串/后缀排序.cpp}
\subsection{字符串哈希}
\lstinputlisting{字符串/字符串哈希.cpp}
\section{图论}
\subsection{最短路}
\subsubsection{Dijkstra朴素版}
\begin{spacing}{1.5}
\input{图论/最短路/Dijkstra朴素版.tex}
\end{spacing}
\lstinputlisting{图论/最短路/Dijkstra朴素版.cpp}
\subsubsection{Dijkstra堆优化版}
\begin{spacing}{1.5}
时间复杂度: $O(m\log n)$
\end{spacing}
\lstinputlisting{图论/最短路/Dijkstra堆优化版.cpp}
\subsubsection{SPFA}
\begin{spacing}{1.5}
最坏时间复杂度 $O(nm)$
\end{spacing}
\lstinputlisting{图论/最短路/SPFA.cpp}
\subsubsection{Bellman-Ford}
\lstinputlisting{图论/最短路/Bellman-Ford.cpp}
\subsection{网络流}
\subsubsection{EK}
\lstinputlisting{图论/网络流/EK.cpp}
\subsubsection{Dinic}
\lstinputlisting{图论/网络流/Dinic.cpp}
\subsubsection{MCMF}
\lstinputlisting{图论/网络流/MCMF.cpp}
\subsection{tarjan}
\subsubsection{tarjan求割点}
\begin{spacing}{1.5}
$low$:最多经过一条后向边能追溯到的最小树中结点编号。

一个顶点$ u $是割点,当且仅当满足(1)或(2):

1. $u $为树根,且$ u $有多于一个子树。因为无向图$ DFS $搜索树中不存在横叉边,所以若有多个子树,这些子树间不会有边相连。

2. $u $不为树根,且满足存在$(u,v)$为树枝边(即 u 为 v 在搜索树中的父亲),并使得$DFN(u)<=Low(v)$.(因为删去$u$后$v$以及$ v $的子树不能到达$ u $的其他子树以及祖先)

求割点时根节点要单独考虑(割边时不需要)。
\end{spacing}
\lstinputlisting{图论/tarjan/tarjan求割点.cpp}
\subsubsection{tarjan求强连通分量}
\begin{spacing}{1.5}
$low$:最多经过一条后向边或栈中横插边所能到达的栈中的最小编号。
\end{spacing}
\lstinputlisting{图论/tarjan/tarjan求强连通分量.cpp}
\subsubsection{tarjan求点双连通分量(圆方树)}
\lstinputlisting{图论/tarjan/tarjan求点双连通分量.cpp}
\subsubsection{tarjan求边双连通分量}
\lstinputlisting{图论/tarjan/tarjan求边双连通分量.cpp}
\subsection{树上问题}
\subsubsection{树的直径}
\lstinputlisting{图论/树上问题/树的直径.cpp}
\subsubsection{树的重心}
\lstinputlisting{图论/树上问题/树的重心.cpp}
\subsubsection{最近公共祖先(倍增)}
\lstinputlisting{图论/树上问题/树上LCA(倍增).cpp}
\subsubsection{树链剖分}
\lstinputlisting{图论/树上问题/树链剖分.cpp}
\subsection{拓扑排序}
\lstinputlisting{图论/topsort.cpp}
\subsection{染色法判断二分图}
\lstinputlisting{图论/染色法判断二分图.cpp}
\subsection{匈牙利算法求最大匹配}
\lstinputlisting{图论/匈牙利算法求最大匹配.cpp}
\subsection{差分约束}
\begin{spacing}{1.5}
求解差分约束系统,有 $m$ 条约束条件,每条都为形如 $x_a-x_b\geq c_k$,$x_a-x_b\leq c_k$ 或 $x_a=x_b$ 的形式,判断该差分约束系统有没有解,如果有解,求出一组解。

\begin{tabular}{ccc}
    题意 & 转化 & 连边 \\
    $x_a - x_b \geq c$ & $x_b - x_a \leq -c$ & add(a, b, -c); \\
    $x_a - x_b \leq c$ & $x_a - x_b \leq c$ & add(b, a, c); \\
    $x_a = x_b$ & $x_a - x_b \leq 0$, $\space x_b - x_a \leq 0$ & add(b, a, 0), add(a, b, 0); \\
  \end{tabular}

若要求出一组解,则每个点到源点的最短路即为解。
具体过程见参考代码
\end{spacing}
\lstinputlisting{图论/差分约束.cpp}
\section{数学}
\subsection{试除法分解质因数}
\lstinputlisting{数学/试除法分解质因数.cpp}
\subsection{欧拉筛}
\lstinputlisting{数学/欧拉筛.cpp}
\subsection{欧拉函数和欧拉定理}
\subsubsection{欧拉函数}
\begin{spacing}{1.5}
欧拉函数定义:$1$ 到 $N$ 中与 $N$ 互质数的个数称为欧拉函数,即 $\varphi(n)=\sum_{i=1}^n[\gcd(n,i)=1]$,记作 $\varphi (N)$ 。

$$\varphi(x)=x\prod_{i=1}^n(1-\frac{1}{p_i})$$

求解单个数的欧拉函数直接质因数分解

欧拉定理:若 $a$ 与 $m$ 互质,则 $a^{\varphi(m)}\equiv 1\pmod m$,变式 $a^b≡a^{b\%\varphi(m)}(mod\ m)$ 。

扩展欧拉定理:若 $a$ 与 $m$ 不互质,则 $a^b\equiv a^{b\% \varphi(m)+\varphi(m)}\pmod m$ 。
$$a^c\equiv\left\{\begin{array}{l|l}a^{c\operatorname{mod}\varphi(m)}&\gcd(a,m)=1\\a^c&\gcd(a,m)\neq1,c<\varphi(m)\\a^{c\operatorname{mod}\varphi(m)+\varphi(m)}&\gcd(a,m)\neq1,c\geq\varphi(m)\end{array}\right.$$
\end{spacing}
\lstinputlisting{数学/欧拉函数和欧拉定理/欧拉函数.cpp}
\subsubsection{线性筛欧拉函数}
\lstinputlisting{数学/欧拉函数和欧拉定理/线性筛欧拉函数.cpp}
\subsection{莫比乌斯函数}
\begin{spacing}{1.5}
设 $n=\prod_{i=1}^mp_i^{c_i}$,则
$$\mu(n)=\begin{cases}1&n=1\\(-1)^m&\prod\limits_{i=1}^mc_i=1(\text{則}c_1=c_2=\cdots=c_m=1)\\0&\text{otherwise}\end{cases}$$
\end{spacing}
\lstinputlisting{数学/莫比乌斯函数.cpp}
\subsection{莫比乌斯函数常见结论}
\begin{spacing}{1.5}
\begin{itemize}
    \item $\sum_{d|n}\mu(d) = [n==1]$
    \item 如果 $g(n)=\sum_{d|n}f(d)$ ,那么 $f(n)=\sum_{d|n}\mu(d)g(\frac{n}{d})$ 。
    \item $[\gcd(x,y)=1] =\sum_{\begin{array}{c}d|\gcd(x,y)\\\end{array}}\mu(d) =\sum_{d=1}\mu(d)[d|x][d|y]$
    \item $\sum_{i=1}^ni[\gcd(i,n)=1]=\frac{\varphi(n)*n}2$
    \item $\mathrm{d}(i*j)=\sum_{x|i}\sum_{y|j}[\gcd(x,y)=1]$
    \item $\varphi*1=\mathrm{Id}$
    \item $\sum_{d|n}\mu(d)=[n=1]$
\end{itemize}

\end{spacing}
\subsection{线性筛约数个数}
\begin{spacing}{1.5}
记 $d_i$ 为 $i$ 的约数个数,$d(i)=\prod_{k=1}^i(a_i+1)$
维护每一个数的最小值因子出现的次数(即 $a_1$ )即可。
\end{spacing}
\lstinputlisting{数学/线性筛约数个数.cpp}
\subsection{线性筛约数和}
\begin{spacing}{1.5}
记 $\sigma(i)$ 表示 $i$ 的约数和
$$\sigma(i)=\prod_{k=1}^i\left(\sum_{a_i=0}^{p_i}p_i^i\right)$$
维护 $\text{low}(i)$ 表示 $i$ 的最小质因子的指数次幂,即 $p^{a_1}_1$,$\text{sum}(i)$ 表示 $i$ 的最小质因子对答案的贡献,即 $\sum_{a_1=0}^{p_1}p_1^1$。
可能会爆int。
\end{spacing}
\lstinputlisting{数学/线性筛约数和.cpp}
\subsection{裴蜀定理}
\begin{spacing}{1.5}
如果$a,b$均为整数,一定存在整数 $x,y$ 使得 $ax+by = gcd(a,b)$ 成立。

推论:对于方程 $ax+by=c$,如果 $gcd(a,b)\mid c$​,则方程一定有解,反之一定无解。
\end{spacing}
\subsection{扩展欧几里得算法}
\begin{spacing}{1.5}
求得的是 $ax+by=gcd(a,b)$ 的一组特解,该方程的通解可以表示为
$$\begin{cases}x'=x+k\frac{b}{gcd(a,b)}\\y'=y-k\frac{a}{gcd(a,b)}\end{cases}(k\in Z)$$
\end{spacing}
\lstinputlisting{数学/扩展欧几里得算法.cpp}
\subsection{快速幂}
\lstinputlisting{数学/快速幂.cpp}
\subsection{BSGS离散对数}
\lstinputlisting{数学/BSGS.cpp}
\subsection{乘法逆元}
\begin{spacing}{1.5}
\begin{itemize}
    \item 当mod为质数时, $ax\equiv 1 \pmod b$由费马小定理有$ax\equiv a^{b-1}\pmod b$, $\therefore x\equiv a^{b-2} \pmod b$, 快速幂求 $a^{b-2}$ 即可
    \item 扩展欧几里得方法(要求 $\gcd(a,b)=1$ ): 等价于求 $ax\equiv 1\pmod p$ 的解,可以写为 $ax+pk = 1$,求解 $x,k$ 即可。
\end{itemize}
\end{spacing}
\lstinputlisting{数学/乘法逆元.cpp}
\subsection{快速递推求逆元}
\begin{spacing}{1.5}
以 $\mathcal O(N)$ 的复杂度完成 $1-N$ 中全部逆元的计算。
\end{spacing}
\lstinputlisting{数学/快速递推求逆元.cpp}
\subsection{中国剩余定理}
\begin{spacing}{1.5}
$x \mod a_i = b_i$
\end{spacing}
\lstinputlisting{数学/中国剩余定理.cpp}
\subsection{高斯消元}
\begin{spacing}{1.5}
求解模意义下n元线性方程组
\end{spacing}
\lstinputlisting{数学/高斯消元.cpp}
\subsection{FFT}
\subsubsection{FFT递归版}
\lstinputlisting{数学/FFT/FFT递归版.cpp}
\subsubsection{FFT迭代版}
\lstinputlisting{数学/FFT/FFT迭代版.cpp}
\subsection{线性基}
\subsubsection{高斯消元法求线性基}
\begin{spacing}{1.5}
高斯消元法构造线性基特点:
\begin{itemize}
    \item 从大到小排列
    \item 各个基的高位没有重复的1
\end{itemize}
\end{spacing}
\lstinputlisting{数学/线性基/高斯消元法.cpp}
\subsubsection{贪心法求线性基}
\begin{spacing}{1.5}
贪心法构造线性基的特点:
\begin{itemize}
    \item 从大到小排列
    \item 各个基的高位可能存在重复的1
    \item 线性基不是唯一的,与原集合即插入顺序有关
\end{itemize}
\end{spacing}
\lstinputlisting{数学/线性基/贪心法.cpp}
\subsection{杜教筛}
\begin{spacing}{1.5}
杜教筛被用于处理一类数论函数的前缀和问题。
求解 $S(n) = \sum_{i=1}^n f(i)$ 的问题,其中 $f(i)$ 是一个数论函数,杜教筛可以在 $O(n^{2/3})$ 的时间复杂度内解决。
找一个数论函数 $g$ 

$$\begin{gathered}\sum_{i=1}^n(f*g)(i)=\sum_{i=1}^n\sum_{d|i}g(d)f\left(\frac id\right)\\
=\sum_{i=1}^n\sum_{j=1}^{\lfloor n/i\rfloor}g(i)f(j) \\
=\sum_{i=1}^ng(i)\sum_{j=1}^{\lfloor n/i\rfloor}f(j) \\
=\sum_{i=1}^ng(i)S\left(\left\lfloor\frac ni\right\rfloor\right) 
\end{gathered}$$

则可以得到递推式:
$$g(1)S(n)=\sum_{i=1}^n(f*g)(i)-\sum_{i=2}^ng(i)S\left(\left\lfloor\frac ni\right\rfloor\right)$$

$g$ 需要满足:
\begin{itemize}
    \item 可以快速计算 $\sum_{i=1}^n (f*g)(i)$
    \item 可以快速计算 $g$ 的前缀和,以用数论分块求解 $\sum_{i=2}^ng(i)S\left(\left\lfloor\frac ni\right\rfloor\right)$
\end{itemize}

时间复杂度:$O(S+\frac{n}{\sqrt{S}})$ ,其中 $S$ 是预处理的 $g$ 的前缀和的规模。取 $S=n^{2/3}$ ,时间复杂度最小为 $O(n^{2/3})$ 。

例如,利用杜教筛求 $\sum_{i=1}^n\mu(i)$ 的前缀和:
利用 $\mu * \mathbf{1} = \mathbf{\varepsilon}$ ,则可以 $g=\mathbf{1}$ ,则有:
$$S(n) = \sum_{i=1}^n{\mu*\mathbf{1}}(i) - \sum_{i=2}^n 1S\left(\left\lfloor\frac ni\right\rfloor\right)\\
=1 - \sum_{i=2}^n S\left(\left\lfloor\frac ni\right\rfloor\right)
$$
\end{spacing}
\lstinputlisting{数学/杜教筛.cpp}
\subsection{min\_25}
\begin{spacing}{1.5}
$g(n, j)$ 表示 $g(n,j)=\sum_{i=1}^nf(i)[i\text{是质数或其最小质因子}>p_j]$ ,转移方程如下:

$$g(n,j)=\begin{cases}g(n,j-1)(P_j^2>n)\\g(n,j-1)-f^{\prime}(P_j)(g(\frac n{P_j},j-1)-\sum_{i=1}^{j-1}f^{\prime}(P_i))(P_j^2\leq n)&\end{cases}$$
求答案:
$$S(n,j)=g(n)-\sum_{i=1}^jf(P_i)+\sum_{k>j}\sum_{e=1}^{P_k^e\leq n}f(P_k^e)[S(\frac n{P_k^e},k)+(e>1)]$$
最后答案为 $S(n,0) + f(1)$,因为 $S$ 函数不会计算到 $1$ 的贡献。 


\end{spacing}
\lstinputlisting{数学/min_25.cpp}
\subsection{数学常见结论}
\subsubsection{插板法}
\begin{spacing}{1.5}
给定 $n$ 个小球 $m$ 个盒子。
\begin{itemize}
    \item 球同,盒不同、不能空,隔板法: $N$ 个小球即一共 $N-1$ 个空,分成 $M$ 堆即 $M-1$ 个隔板,答案为 $\dbinom{n-1}{m-1}$ 。
    \item 球同,盒不同、能空,隔板法:多出 $M-1$ 个虚空球,答案为 $\dbinom{m-1+n}{n}$ 。
    \item 球同,盒同、能空:$\dfrac{1}{(1-x)(1-x^2)\dots(1-x^m)}$ 的 $x^n$ 项的系数。动态规划,答案为 $$
dp[i][j]= \left\{\begin{array}{ll}
dp[i][j-1]+dp[i-j][j] & \text{if } i \geq j \\
dp[i][j-1] & \text{if } i < j \\
1 & \text{if } j = 1 \ \text{or}\ i \leq 1
\end{array}\right.
$$


    \item 球同,盒同、不能空:$\dfrac{x^m}{(1-x)(1-x^2)\dots(1-x^m)}$ 的 $x^n$ 项的系数。动态规划,答案为 $$
dp[n][m]= \left\{
\begin{array}{ll}
dp[n-m][m] & \text{if } n \ge m \\
0 & \text{if } n < m \\
\end{array}
\right.
$$

    \item 球不同,盒同、不能空:第二类斯特林数 ${\tt Stirling2}(n,m)$ ,答案为 $$
dp[n][m] = \left\{
\begin{array}{ll}
m \cdot dp[n-1][m] + dp[n-1][m-1] & \text{if } 1 \le m < n \\
1 & \text{if } 0 \le n = m \\
0 & \text{if } m = 0 \text{ and } 1 \le n \\
\end{array}
\right.
$$

    \item 球不同,盒同、能空:第二类斯特林数之和 $\displaystyle\sum_{i=1}^m{\tt Stirling2}(n,m)$ ,答案为 $\sum_{i = 0}^{m} dp[n][i]$ 。
    \item 球不同,盒不同、不能空:第二类斯特林数乘上 $m$ 的阶乘 $m!\cdot{\tt Stirling2}(n,m)$ ,答案为 $dp[n][m] * m!$ 。
    \item 球不同,盒不同、能空:答案为 $m^n$ 。
\end{itemize}
\end{spacing}
\subsubsection{组合数学常见性质}
\begin{spacing}{1.5}
\begin{itemize}
    \item $k *C^k_n=n*C^{k-1}_{n-1}$ ;
    \item $C_k^n*C_m^k=C_m^n*C_{m-n}^{m-k}$ ;
    \item $C_n^k+C_n^{k+1}=C_{n+1}^{k+1}$ ;
    \item $\sum_{i=0}^n C_n^i=2^n$ ;
    \item $\sum_{k=0}^n(-1)^k*C_n^k=0$ 。
    \item 二项式反演:$\left\{\begin{matrix} \displaystyle f_n=\sum_{i=0}^n{n\choose i}g_i\Leftrightarrow g_n=\sum_{i=0}^n(-1)^{n-i}{n\choose i}f_i \\
                  \displaystyle f_k=\sum_{i=k}^n{i\choose k}g_i\Leftrightarrow g_k=\sum_{i=k}^n(-1)^{i-k}{i\choose k}f_i\end{matrix}\right. $ ;
    \item $\displaystyle \sum_{i=1}^{n}i{n\choose i}=n * 2^{n-1}$ ;
    \item $\displaystyle \sum_{i=1}^{n}i^2{n\choose i}=n*(n+1)*2^{n-2}$ ;
    \item $\displaystyle \sum_{i=1}^{n}\dfrac{1}{i}{n\choose i}=\sum_{i=1}^{n}\dfrac{1}{i}$ ;
    \item $\displaystyle \sum_{i=0}^{n}{n\choose i}^2={2n\choose n}$ ;
    \item 拉格朗日恒等式:$\displaystyle \sum_{i=1}^{n}\sum_{j=i+1}^{n}(a_ib_j-a_jb_i)^2=(\sum_{i=1}^{n}a_i)^2(\sum_{i=1}^{n}b_i)^2-(\sum_{i=1}^{n}a_ib_i)^2$ 。
\end{itemize}
\end{spacing}
\subsubsection{斯特林数}
\begin{spacing}{1.5}

\textbf{第二类斯特林数(斯特林子集数)}
$\begin{Bmatrix}n\\ k\end{Bmatrix}$,也可记做 $S(n,k)$,表示将 $n$ 个两两不同的元素,划分为 $k$ 个互不区分的非空子集的方案数。
递推式
$$\begin{Bmatrix}n\\ k\end{Bmatrix}=\begin{Bmatrix}n-1\\ k-1\end{Bmatrix}+k\begin{Bmatrix}n-1\\ k\end{Bmatrix}$$
边界是
$\begin{Bmatrix}n\\ 0\end{Bmatrix}=[n=0]。$

考虑用组合意义来证明。
我们插入一个新元素时,有两种方案:
\begin{itemize}
    \item 将新元素单独放入一个子集,有 $\begin{Bmatrix}n-1\\ k-1\end{Bmatrix}$ 种方案;
    \item 将新元素放入一个现有的非空子集,有 $k\begin{Bmatrix}n-1\\ k\end{Bmatrix}$ 种方案。
\end{itemize}
根据加法原理,将两式相加即可得到递推式。

\textbf{第一类斯特林数(斯特林轮换数) }
$\begin{bmatrix}n\\ k\end{bmatrix}$,也可记做 $s(n,k)$,表示将 $n$ 个两两不同的元素,划分为 $k$ 个互不区分的非空轮换的方案数。

一个轮换就是一个首尾相接的环形排列。我们可以写出一个轮换 $[A,B,C,D]$,并且我们认为 $[A,B,C,D]=[B,C,D,A]=[C,D,A,B]=[D,A,B,C]$,即,两个可以通过旋转而互相得到的轮换是等价的。注意,我们不认为两个可以通过翻转而相互得到的轮换等价,即 $[A,B,C,D]\neq[D,C,B,A]$。
递推式
$$\begin{bmatrix}n\\ k\end{bmatrix}=\begin{bmatrix}n-1\\ k-1\end{bmatrix}+(n-1)\begin{bmatrix}n-1\\ k\end{bmatrix}$$
边界是
$\begin{bmatrix}n\\ 0\end{bmatrix}=[n = 0]$。

该递推式的证明可以考虑其组合意义。我们插入一个新元素时,有两种方案:

\begin{itemize}
    \item 将该新元素置于一个单独的轮换中,共有 $\begin{bmatrix}n-1\\ k-1\end{bmatrix}$ 种方案;
    \item 将该元素插入到任何一个现有的轮换中,共有 $(n-1)\begin{bmatrix}n-1\\ k\end{bmatrix}$ 种方案。
\end{itemize}

根据加法原理,将两式相加即可得到递推式。
\end{spacing}
\subsubsection{卡特兰数}
\begin{spacing}{1.5}
是一类奇特的组合数,前几项为 $1,1,2,5,14,42,132,429,1430,4862$ 。如遇到以下问题,则直接套用即可。

\begin{itemize}
    \item 【括号匹配问题】 $n$ 个左括号和 $n$ 个右括号组成的合法括号序列的数量,为 $Cat_n$ 。
    \item 【进出栈问题】 $1,2,…,n$ 经过一个栈,形成的合法出栈序列的数量,为 $Cat_n$ 。
    \item 【二叉树生成问题】 $n$ 个节点构成的不同二叉树的数量,为 $Cat_n$ 。
    \item 【路径数量问题】在平面直角坐标系上,每一步只能**向上**或**向右**走,从 $(0,0)$ 走到 $(n,n)$ ,并且除两个端点外不接触直线 $y=x$ 的路线数量,为 $2Cat_{n-1}$ 。
\end{itemize}

计算公式:$Cat_n=\dfrac{C^n_{2n}}{n+1}$ ,$C_n=\dfrac{C_{n-1}*(4n-2)}{n+1}$ 。
\end{spacing}
\subsubsection{斐波那契数列}
\begin{spacing}{1.5}
通项公式:$F_n=\dfrac{1}{\sqrt 5}*  \Big[ \Big( \dfrac{1+\sqrt 5}{2} \Big)^n - \Big( \dfrac{1-\sqrt 5}{2} \Big)^n \Big]$ 。

直接结论:
\begin{itemize}
    \item 卡西尼性质:$F_{n-1} * F_{n+1}-F_n^2=(-1)^n$ ;
    \item $F_{n}^2+F_{n+1}^2=F_{2n+1}$ ;
    \item $F_{n+1}^2-F_{n-1}^2=F_{2n}$ (由上一条写两遍相减得到);
    \item 若存在序列 $a_0=1,a_n=a_{n-1}+a_{n-3}+a_{n-5}+...(n\ge 1)$ 则 $a_n=F_n(n\ge 1)$ ;
    \item 齐肯多夫定理:任何正整数都可以表示成若干个不连续的斐波那契数( $F_2$ 开始)可以用贪心实现。
\end{itemize}

求和公式结论:
\begin{itemize}
    \item 奇数项求和:$F_1+F_3+F_5+...+F_{2n-1}=F_{2n}$ ;
    \item 偶数项求和:$F_2+F_4+F_6+...+F_{2n}=F_{2n+1}-1$ ;
    \item 平方和:$F_1^2+F_2^2+F_3^2+...+F_n^2=F_n*F_{n+1}$ ;
    \item $F_1+2F_2+3F_3+...+nF_n=nF_{n+2}-F_{n+3}+2$ ;
    \item $-F_1+F_2-F_3+...+(-1)^nF_n=(-1)^n(F_{n+1}-F_n)+1$ ;
    \item $F_{2n-2m-2}(F_{2n}+F_{2n+2})=F_{2m+2}+F_{4n-2m}$ 。
\end{itemize}

数论结论:
\begin{itemize}
    \item $F_a \mid F_b \Leftrightarrow a \mid b$ ;
    \item $\gcd(F_a,F_b)=F_{\gcd(a,b)}$ ;
    \item 当 $p$ 为 $5k\pm 1$ 型素数时,$\begin{cases} F_{p-1}\equiv 0\pmod p \\ F_p\equiv 1\pmod p \\ F_{p+1}\equiv 1\pmod p \end{cases}$ ;
    \item 当 $p$ 为 $5k\pm 2$ 型素数时,$\begin{cases} F_{p-1}\equiv 1\pmod p \\ F_p\equiv -1\pmod p \\ F_{p+1}\equiv 0\pmod p \end{cases}$ ;
    \item $F(n)\%m$ 的周期 $\le 6m$ ( $m=2\times 5^k$ 时取到等号);
    \item 既是斐波那契数又是平方数的有且仅有 $1,144$ 。
\end{itemize}
\end{spacing}
\section{博弈论}
\subsection{巴什博弈}
\subsubsection{朴素巴什博弈}
\begin{spacing}{1.5}
有 $N$ 个石子,两名玩家轮流行动,按以下规则取石子:

规定:每人每次可以取走 $X(1 \le X \le M)$ 个石子,拿到最后一颗石子的一方获胜。

双方均采用最优策略,询问谁会获胜。

两名玩家轮流报数。

规定:第一个报数的人可以报 $X(1 \le X \le M)$ ,后报数的人需要比前者所报数大 $Y(1 \le Y \le M)$ ,率先报到 $N$ 的人获胜。

双方均采用最优策略,询问谁会获胜。

\begin{itemize}
    \item $N=K\cdot(M+1)$ (其中 $K \in \mathbb{N}^+$ ),后手必胜(后手可以控制每一回合结束时双方恰好取走 $M+1$ 个,重复 $K$ 轮后即胜利);
    \item $N=K\cdot(M+1)+R$ (其中 $K \in \mathbb{N}^+,0 < R < M + 1$ ),先手必胜(先手先取走 $R$ 个,之后控制每一回合结束时双方恰好取走 $M+1$ 个,重复 $K$ 轮后即胜利)。
\end{itemize}
\end{spacing}
\subsubsection{扩展巴什博弈}
\begin{spacing}{1.5}
有 $N$ 颗石子,两名玩家轮流行动,按以下规则取石子:。

规定:每人每次可以取走 $X(a \le X \le b)$ 个石子,如果最后剩余物品的数量小于 $a$ 个,则不能再取,拿到最后一颗石子的一方获胜。

双方均采用最优策略,询问谁会获胜。

\begin{itemize}
    \item $N = K\cdot(a+b)$ 时,后手必胜;
    \item $N = K\cdot(a+b)+R_1$ (其中 $K \in \mathbb{N}^+,0 < R_1 < a$ ) 时,后手必胜(这些数量不够再取一次,先手无法逆转局面);
    \item $N = K\cdot(a+b)+R_2$ (其中 $K \in \mathbb{N}^+,a \le R_2 \le b$ ) 时,先手必胜;
    \item $N = K\cdot(a+b)+R_3$ (其中 $K \in \mathbb{N}^+,b < R_3 < a + b$ ) 时,先手必胜(这些数量不够再取一次,后手无法逆转局面)。
\end{itemize}
\end{spacing}
\subsection{Nim博弈}
\subsubsection{Nim博弈}
\begin{spacing}{1.5}
有 $N$ 堆石子,给出每一堆的石子数量,两名玩家轮流行动,按以下规则取石子:

规定:每人每次任选一堆,取走正整数颗石子,拿到最后一颗石子的一方获胜(注:几个特点是\textbf{不能跨堆}、\textbf{不能不拿})。

双方均采用最优策略,询问谁会获胜。

记初始时各堆石子的数量 $(A_1,A_2,\dots,A_n)$ ,定义尼姆和 $Sum_N = A_1 \oplus A_2 \oplus \dots \oplus A_n$ 。

当 $\pmb{ Sum_N = 0 }$ 时先手必败,反之先手必胜。

\textbf{具体取法}

先计算出尼姆和,再对每一堆石子计算 $A_i \oplus Sum_N$ ,记为 $X_i$ 。

若得到的值 $X_i<A_i$ ,$X_i$ 即为一个可行解,即剩下 $\pmb X_i$ 颗石头,取走 $\pmb {A_i - X_i}$ 颗石头(这里取小于号是因为至少要取走 $1$ 颗石子)。

\end{spacing}
\subsubsection{Nim\_K}
\begin{spacing}{1.5}
有 $N$ 堆石子,给出每一堆的石子数量,两名玩家轮流行动,按以下规则取石子:

规定:每人每次任选不超过 $K$ 堆,对每堆都取走不同的正整数颗石子,拿到最后一颗石子的一方获胜。

双方均采用最优策略,询问谁会获胜。

把每一堆石子的石子数用二进制表示,定义 $One_i$ 为二进制第 $i$ 位上 $1$ 的个数。

\textbf{以下局面先手必胜:}

对于每一位, $\pmb{One_1,One_2,\dots ,One_N}$ 均不为 $\pmb{K+1}$ 的倍数。

\end{spacing}
\subsubsection{反Nim游戏}
\begin{spacing}{1.5}
有 $N$ 堆石子,给出每一堆的石子数量,两名玩家轮流行动,按以下规则取石子:

规定:每人每次任选一堆,取走正整数颗石子,拿到最后一颗石子的一方**出局**。

双方均采用最优策略,询问谁会获胜。

\begin{itemize}
    \item 所有堆的石头数量均不超过 $1$ ,且 $\pmb {Sum_N=0}$ (也可看作“且有偶数堆”)时;
    \item 至少有一堆的石头数量大于 $1$ ,且 $\pmb{Sum_N \neq 0}$ 。
\end{itemize}

\end{spacing}
\subsection{SG游戏}
\subsubsection{SG定理和SG函数}
\begin{spacing}{1.5}
我们使用以下几条规则来定义暴力求解的过程:

\begin{itemize}
    \item 使用数字来表示输赢情况,$0$ 代表局面必败,非 $0$ 代表**存在必胜可能**,我们称这个数字为这个局面的SG值;
    \item 找到最终态,根据题意人为定义最终态的输赢情况;
    \item 对于非最终态的某个节点,其SG值为所有子节点的SG值取 $\tt{}mex$ ;
    \item 单个游戏的输赢态即对应根节点的SG值是否为 $0$ ,为 $0$ 代表先手必败,非 $0$ 代表先手必胜;
    \item 多个游戏的总SG值为单个游戏SG值的异或和。
\end{itemize}

使用哈希表,以 $\mathcal{O} (N + M)$ 的复杂度计算。

\end{spacing}
\lstinputlisting{博弈论/SG游戏/SG函数.cpp}
\subsubsection{反SG博弈}
\begin{spacing}{1.5}
SG 游戏中最先不能行动的一方获胜。

\textbf{以下局面先手必胜:}

\begin{itemize}
    \item 单局游戏的SG值均不超过 $\pmb 1$ ,且总SG值为 $\pmb 0$;
    \item 至少有一局单局游戏的SG值大于 $\pmb 1$ ,且总SG值不为 $\pmb 0$ 。
\end{itemize}

在本质上,这与 Anti-Nim 游戏的结论一致。

\end{spacing}
\section{计算几何}
\subsection{计算几何}
\lstinputlisting{计算几何/计算几何.cpp}
\end{document}