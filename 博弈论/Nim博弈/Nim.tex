有 $N$ 堆石子,给出每一堆的石子数量,两名玩家轮流行动,按以下规则取石子:

规定:每人每次任选一堆,取走正整数颗石子,拿到最后一颗石子的一方获胜(注:几个特点是\textbf{不能跨堆}、\textbf{不能不拿})。

双方均采用最优策略,询问谁会获胜。

记初始时各堆石子的数量 $(A_1,A_2,\dots,A_n)$ ,定义尼姆和 $Sum_N = A_1 \oplus A_2 \oplus \dots \oplus A_n$ 。

当 $\pmb{ Sum_N = 0 }$ 时先手必败,反之先手必胜。

\textbf{具体取法}

先计算出尼姆和,再对每一堆石子计算 $A_i \oplus Sum_N$ ,记为 $X_i$ 。

若得到的值 $X_i<A_i$ ,$X_i$ 即为一个可行解,即剩下 $\pmb X_i$ 颗石头,取走 $\pmb {A_i - X_i}$ 颗石头(这里取小于号是因为至少要取走 $1$ 颗石子)。
