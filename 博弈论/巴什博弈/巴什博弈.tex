有 $N$ 个石子,两名玩家轮流行动,按以下规则取石子:

规定:每人每次可以取走 $X(1 \le X \le M)$ 个石子,拿到最后一颗石子的一方获胜。

双方均采用最优策略,询问谁会获胜。

两名玩家轮流报数。

规定:第一个报数的人可以报 $X(1 \le X \le M)$ ,后报数的人需要比前者所报数大 $Y(1 \le Y \le M)$ ,率先报到 $N$ 的人获胜。

双方均采用最优策略,询问谁会获胜。

\begin{itemize}
    \item $N=K\cdot(M+1)$ (其中 $K \in \mathbb{N}^+$ ),后手必胜(后手可以控制每一回合结束时双方恰好取走 $M+1$ 个,重复 $K$ 轮后即胜利);
    \item $N=K\cdot(M+1)+R$ (其中 $K \in \mathbb{N}^+,0 < R < M + 1$ ),先手必胜(先手先取走 $R$ 个,之后控制每一回合结束时双方恰好取走 $M+1$ 个,重复 $K$ 轮后即胜利)。
\end{itemize}