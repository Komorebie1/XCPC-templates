有 $N$ 颗石子,两名玩家轮流行动,按以下规则取石子:。

规定:每人每次可以取走 $X(a \le X \le b)$ 个石子,如果最后剩余物品的数量小于 $a$ 个,则不能再取,拿到最后一颗石子的一方获胜。

双方均采用最优策略,询问谁会获胜。

\begin{itemize}
    \item $N = K\cdot(a+b)$ 时,后手必胜;
    \item $N = K\cdot(a+b)+R_1$ (其中 $K \in \mathbb{N}^+,0 < R_1 < a$ ) 时,后手必胜(这些数量不够再取一次,先手无法逆转局面);
    \item $N = K\cdot(a+b)+R_2$ (其中 $K \in \mathbb{N}^+,a \le R_2 \le b$ ) 时,先手必胜;
    \item $N = K\cdot(a+b)+R_3$ (其中 $K \in \mathbb{N}^+,b < R_3 < a + b$ ) 时,先手必胜(这些数量不够再取一次,后手无法逆转局面)。
\end{itemize}