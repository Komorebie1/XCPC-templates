在 Tarjan 算法中为每个结点 u 维护了以下几个变量:
\begin{enumerate}
    \item $\textit{dfn}_u$:深度优先搜索遍历时结点 u 被搜索的次序。
    \item $\textit{low}_u$:在 u 的子树中能够回溯到的最早的已经在栈中的结点。设以 u 为根的子树为 $\textit{Subtree}_u$。$\textit{low}_u$ 定义为以下结点的 $\textit{dfn}$ 的最小值:$\textit{Subtree}_u$ 中的结点;从 $\textit{Subtree}_u$ 通过一条不在搜索树上的边能到达的结点。
\end{enumerate}

一个结点的子树内结点的 dfn 都大于该结点的 dfn。

从根开始的一条路径上的 dfn 严格递增,low 严格非降。

按照深度优先搜索算法搜索的次序对图中所有的结点进行搜索,维护每个结点的 dfn 与 low 变量,且让搜索到的结点入栈。每当找到一个强连通元素,就按照该元素包含结点数目让栈中元素出栈。在搜索过程中,对于结点 u 和与其相邻的结点 v(v 不是 u 的父节点)考虑 3 种情况:

\begin{enumerate}
    \item v 未被访问:继续对 v 进行深度搜索。在回溯过程中,用 $\textit{low}_v$ 更新 $\textit{low}_u$。因为存在从 u 到 v 的直接路径,所以 v 能够回溯到的已经在栈中的结点,u 也一定能够回溯到。
    \item v 被访问过,已经在栈中:根据 low 值的定义,用 $\textit{dfn}_v$ 更新 $\textit{low}_u$。
    \item v 被访问过,已不在栈中:说明 v 已搜索完毕,其所在连通分量已被处理,所以不用对其做操作。
\end{enumerate}

$low$:最多经过一条后向边或栈中横插边所能到达的栈中的最小编号。