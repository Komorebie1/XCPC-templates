
\textbf{第二类斯特林数(斯特林子集数)}
$\begin{Bmatrix}n\\ k\end{Bmatrix}$,也可记做 $S(n,k)$,表示将 $n$ 个两两不同的元素,划分为 $k$ 个互不区分的非空子集的方案数。
递推式
$$\begin{Bmatrix}n\\ k\end{Bmatrix}=\begin{Bmatrix}n-1\\ k-1\end{Bmatrix}+k\begin{Bmatrix}n-1\\ k\end{Bmatrix}$$
边界是
$\begin{Bmatrix}n\\ 0\end{Bmatrix}=[n=0]。$

考虑用组合意义来证明。
我们插入一个新元素时,有两种方案:
\begin{itemize}
    \item 将新元素单独放入一个子集,有 $\begin{Bmatrix}n-1\\ k-1\end{Bmatrix}$ 种方案;
    \item 将新元素放入一个现有的非空子集,有 $k\begin{Bmatrix}n-1\\ k\end{Bmatrix}$ 种方案。
\end{itemize}
根据加法原理,将两式相加即可得到递推式。

\textbf{第一类斯特林数(斯特林轮换数) }
$\begin{bmatrix}n\\ k\end{bmatrix}$,也可记做 $s(n,k)$,表示将 $n$ 个两两不同的元素,划分为 $k$ 个互不区分的非空轮换的方案数。

一个轮换就是一个首尾相接的环形排列。我们可以写出一个轮换 $[A,B,C,D]$,并且我们认为 $[A,B,C,D]=[B,C,D,A]=[C,D,A,B]=[D,A,B,C]$,即,两个可以通过旋转而互相得到的轮换是等价的。注意,我们不认为两个可以通过翻转而相互得到的轮换等价,即 $[A,B,C,D]\neq[D,C,B,A]$。
递推式
$$\begin{bmatrix}n\\ k\end{bmatrix}=\begin{bmatrix}n-1\\ k-1\end{bmatrix}+(n-1)\begin{bmatrix}n-1\\ k\end{bmatrix}$$
边界是
$\begin{bmatrix}n\\ 0\end{bmatrix}=[n = 0]$。

该递推式的证明可以考虑其组合意义。我们插入一个新元素时,有两种方案:

\begin{itemize}
    \item 将该新元素置于一个单独的轮换中,共有 $\begin{bmatrix}n-1\\ k-1\end{bmatrix}$ 种方案;
    \item 将该元素插入到任何一个现有的轮换中,共有 $(n-1)\begin{bmatrix}n-1\\ k\end{bmatrix}$ 种方案。
\end{itemize}

根据加法原理,将两式相加即可得到递推式。